\chapter{기저와 차원}

\chapterintro{기저와 차원은 벡터공간의 크기를 측정하는 핵심 도구입니다. 이 장에서는 기저의 존재와 차원의 불변성을 증명해 선형대수의 기초 언어를 완성하겠습니다.}
\chaptergoals{기저의 의미를 생성과 독립의 결합으로 설명할 수 있다.}{차원 정의의 정당성을 증명할 수 있다.}{차원공식을 활용해 공간 구조를 해석할 수 있다.}

\sectiontemplate{기저의 정의와 존재}{기저는 좌표계를 만드는 최소 데이터입니다. 생성과 독립의 균형으로 정의됩니다.}{\item 기저를 엄밀히 정의할 수 있습니다.\item 유한 생성공간에서 기저를 구성할 수 있습니다.\item 기저의 비유일성을 이해할 수 있습니다.}{\item 생성\item 일차독립}

\begin{definition}
벡터공간 $V$의 부분집합 $B$가 $V$를 생성하고 일차독립이면 $B$를 $V$의 \emph{기저}라 한다.
\end{definition}

\paragraph{증명 전략.}
유한 생성집합에서 종속 벡터를 하나씩 제거하면 독립인 생성집합, 즉 기저를 얻을 수 있습니다.

\begin{theorem}
유한 생성 벡터공간은 기저를 가진다.
\end{theorem}

\begin{proof}
$V=\operatorname{span}(S)$인 유한집합 $S$를 잡는다. $S$가 독립이면 끝이다. 종속이면 어떤 원소가 나머지의 선형결합이므로 제거해도 생성성은 유지된다. 이 과정을 유한 번 반복하면 독립인 생성집합을 얻고, 이것이 기저다.
\end{proof}

\paragraph{의미 해석.}
기저 존재는 "모든 유한차원 선형대수 계산은 좌표계로 환원 가능"하다는 선언입니다.

\subsection*{주의}
\begin{warning}
기저는 일반적으로 유일하지 않습니다. 기저의 개수(원소 수)만이 불변입니다.
\end{warning}
\begin{warning}
무한차원 공간에서는 유한한 기저가 존재하지 않을 수 있습니다.
\end{warning}

\subsection*{자가진단퀴즈}
\begin{enumerate}[label=\arabic*.]
\item $\mathbb{R}^3$의 서로 다른 기저 두 개를 써 보십시오.
\item 종속 벡터 제거가 왜 생성성을 유지하는지 설명하십시오.
\item 다항식공간 $P_2$의 기저를 하나 제시하십시오.
\end{enumerate}

\sectiontemplate{차원의 정의와 차원공식}{차원은 기저의 원소 수입니다. 모든 기저의 크기가 같다는 정당화가 핵심입니다.}{\item 차원을 엄밀히 정의할 수 있습니다.\item 모든 기저의 크기가 같음을 설명할 수 있습니다.\item 합공간 차원공식을 사용할 수 있습니다.}{\item 기저 존재\item 독립/생성 비교}

\paragraph{증명 전략.}
교체정리를 사용하면 독립집합 크기는 생성집합 크기를 넘지 못한다는 사실을 얻고, 이를 양방향으로 적용하면 기저 크기 불변성이 따라옵니다.

\begin{theorem}[차원의 well-definedness]
유한차원 벡터공간 $V$의 임의의 두 기저는 같은 원소 수를 가진다.
\end{theorem}

\begin{proof}
기저 $B_1,B_2$를 잡는다. $B_1$은 독립, $B_2$는 생성집합이므로 교체정리에 의해 $|B_1|\le |B_2|$이다. 반대로도 적용하면 $|B_2|\le |B_1|$이다. 따라서 $|B_1|=|B_2|$.
\end{proof}

\paragraph{의미 해석.}
이 정리 덕분에 차원은 기저 선택과 무관한 공간의 고유량이 됩니다.

\begin{theorem}[합공간 차원공식]
부분공간 $U,W\le V$에 대해
$$
\dim(U+W)=\dim U+\dim W-\dim(U\cap W)
$$
가 성립한다.
\end{theorem}

\begin{proof}
$U\cap W$의 기저를 $\{z_1,\dots,z_k\}$로 잡고 이를 각각 $U,W$의 기저로 확장한다. 확장된 기저들을 합치면 $U+W$의 기저가 되며, 원소 수를 세면 식이 나온다.
\end{proof}

\begin{example}
$\mathbb{R}^3$에서 두 평면이 한 직선에서 만나면 $\dim(U\cap W)=1$, 따라서 $\dim(U+W)=2+2-1=3$입니다.
\end{example}

\subsection*{주의}
\begin{warning}
$\dim(U+W)=\dim U+\dim W$는 일반적으로 거짓입니다. 교집합 차원을 반드시 빼야 합니다.
\end{warning}
\begin{warning}
직합 판정은 $U\cap W=\{0\}$ 조건을 확인해야 합니다.
\end{warning}

\subsection*{자가진단퀴즈}
\begin{enumerate}[label=\arabic*.]
\item 차원공식을 이용해 두 부분공간의 교집합 차원을 계산해 보십시오.
\item 왜 기저 크기 불변성이 차원 정의의 핵심인지 설명하십시오.
\item $V=U\oplus W$일 때 $\dim V=\dim U+\dim W$를 증명하십시오.
\end{enumerate}

\chapterapplicationtemplate{해공간과 열공간의 차원을 비교하여 연립방정식의 자유도를 차원공식으로 설명하는 예제를 구성해 보십시오.}

\chaptersummarytemplate
\chapterexercisestemplate
