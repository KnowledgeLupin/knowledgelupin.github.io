% !TEX program = xelatex
% !TEX TS-program = xelatex

\newif\ifchapterstandalone
\ifdefined\chapter
  \chapterstandalonefalse
\else
  \chapterstandalonetrue
\fi

\let\chapterstandaloneend\relax
\ifchapterstandalone
  \documentclass[11pt,a4paper]{book}
  \InputIfFileExists{../preamble.tex}{}{
    \InputIfFileExists{preamble.tex}{}{
      \InputIfFileExists{latex/linear-algebra/preamble.tex}{}{
        \errmessage{preamble.tex not found}
      }
    }
  }
  \begin{document}
  \mainmatter
  \def\chapterstandaloneend{\end{document}}
\fi

\chapter{벡터공간}

\section{체(Field)와 벡터공간의 공리}

선형대수의 문은 대개 계산으로 열린다. 우리는 \(A\vect{x}=\vect{b}\) 같은 식 앞에 서서 해를 구하고, 행을 바꾸고, 소거하고, 다시 정리한다. 처음에는 이 모든 과정이 기법의 문제처럼 보인다. 그러나 계산을 조금 더 오래 붙들고 있으면, 기술보다 먼저 물어야 할 질문이 드러난다. 우리는 지금 어떤 대상에서, 어떤 규칙으로 계산하고 있는가.

독자는 이미 \(\R^n\)에서 벡터를 더하고 실수를 곱하는 연산에 익숙하다. 그런데 선형대수의 시야는 거기서 멈추지 않는다. 다항식도, 함수도, 수열도 같은 방식으로 다룰 수 있다면, 서로 다른 대상을 관통하는 공통의 틀이 있다는 뜻이다. 벡터공간은 바로 그 틀을 이름 붙인 개념이다. 모양이 아니라 구조를 본다는 것, 그것이 선형대수에서 말하는 추상화의 첫걸음이다.

이제 관심은 “무엇을 계산하느냐”보다 “어떤 공리가 허용되느냐”로 옮겨간다. 덧셈과 스칼라곱이 만족해야 할 최소한의 규칙만 정해두면, 대상이 달라져도 논리의 틀은 그대로 적용된다. 그래서 공리에서 출발한 증명은 특정 예시에 기대지 않고 구조 자체에 근거한다.

실제로 이 절에서 보게 되듯, 공리만으로도 영벡터와 역벡터의 유일성, 소거법칙, \(0\vect{v}=\vect{0}\), \(a\vect{0}=\vect{0}\), \((-1)\vect{v}=-\vect{v}\) 같은 기본 성질이 따라 나온다. 따라서, 이 절의 목표는 정의를 나열하는 데 있지 않다. 이후의 기저, 차원, 선형사상, 고유값 이론의 논리적 기반을 분명히 하는 데 있다.

\begin{definition}[체]
집합 $K$와 연산 $+$, $\cdot$가 다음을 만족하면 $K$를 \textbf{체}라 한다.
\begin{enumerate}[label=(\arabic*)]
  \item $(K,+)$는 아벨군이다.
  \item $(K\setminus\{0\},\cdot)$는 아벨군이다.
  \item 분배법칙 $a(b+c)=ab+ac$가 성립한다.
\end{enumerate}
\end{definition}

대표적인 예는 $\Q,\R,\C$이며, $\Z$는 체가 아니다. 예를 들어 $2\in\Z$의 곱셈 역원 $\frac{1}{2}$는 $\Z$에 속하지 않는다.

\begin{definition}[벡터공간]
체 $K$ 위의 집합 $V$에 대해
\begin{itemize}
  \item 벡터 덧셈 $V\times V\to V$, $(\vect{u},\vect{v})\mapsto \vect{u}+\vect{v}$,
  \item 스칼라곱 $K\times V\to V$, $(a,\vect{v})\mapsto a\vect{v}$
\end{itemize}
가 정의되어 있고, 임의의 $a,b\in K$, $\vect{u},\vect{v},\vect{w}\in V$에 대해 다음 공리가 성립하면 $V$를 $K$ 위의 \textbf{벡터공간}이라 한다.
\begin{enumerate}[label=(VS\arabic*)]
  \item $(\vect{u}+\vect{v})+\vect{w}=\vect{u}+(\vect{v}+\vect{w})$
  \item $\vect{u}+\vect{v}=\vect{v}+\vect{u}$
  \item 영벡터 $\vect{0}\in V$가 존재하여 $\vect{v}+\vect{0}=\vect{v}$
  \item 각 $\vect{v}\in V$에 대해 $\vect{v}+(-\vect{v})=\vect{0}$인 $-\vect{v}$가 존재
  \item $a(\vect{u}+\vect{v})=a\vect{u}+a\vect{v}$
  \item $(a+b)\vect{v}=a\vect{v}+b\vect{v}$
  \item $(ab)\vect{v}=a(b\vect{v})$
  \item $1\vect{v}=\vect{v}$
\end{enumerate}
\end{definition}

\begin{theorem}[영벡터와 역벡터의 유일성]
$K$ 위의 벡터공간 $V$에서 영벡터는 유일하고, 각 벡터의 덧셈 역벡터도 유일하다.
\end{theorem}

\begin{proof}
먼저 영벡터의 유일성을 보이자. $\vect{0},\vect{0}'$가 모두 영벡터라고 가정하면,
\[
\vect{0}+\vect{0}'=\vect{0}'\quad\text{및}\quad \vect{0}+\vect{0}'=\vect{0}
\]
가 되어 $\vect{0}=\vect{0}'$를 얻는다.

다음으로 역벡터의 유일성을 보이자. $\vect{w},\vect{w}'$가 모두 $\vect{v}$의 역벡터라면
$\vect{v}+\vect{w}=\vect{0}$, $\vect{v}+\vect{w}'=\vect{0}$이고,
\[
\vect{w}=\vect{w}+\vect{0}=\vect{w}+(\vect{v}+\vect{w}')
=(\vect{w}+\vect{v})+\vect{w}'=(\vect{v}+\vect{w})+\vect{w}'
=\vect{0}+\vect{w}'=\vect{w}'
\]
이므로 역벡터도 유일하다.
\end{proof}

\begin{theorem}[소거법칙]
$K$ 위의 벡터공간 $V$에서 임의의 $\vect{u},\vect{v},\vect{w}\in V$에 대해
\[
\vect{u}+\vect{w}=\vect{v}+\vect{w}
\]
이면 $\vect{u}=\vect{v}$이다.
\end{theorem}

\begin{proof}
가정한 식의 양변에 $-\vect{w}$를 더하면
\[
(\vect{u}+\vect{w})+(-\vect{w})=(\vect{v}+\vect{w})+(-\vect{w})
\]
이고 결합법칙과 역벡터의 정의에 의해
\[
\vect{u}+\vect{0}=\vect{v}+\vect{0}
\]
을 얻는다. 따라서 $\vect{u}=\vect{v}$이다.
\end{proof}

\begin{theorem}[기본 계산 법칙]
임의의 $a\in K$, $\vect{v}\in V$에 대해 다음이 성립한다.
\begin{enumerate}[label=(\arabic*)]
  \item $0\vect{v}=\vect{0}$
  \item $a\vect{0}=\vect{0}$
  \item $(-1)\vect{v}=-\vect{v}$
  \item $(-a)\vect{v}=-(a\vect{v})$
\end{enumerate}
\end{theorem}

\begin{proof}
(1) $(0+0)\vect{v}=0\vect{v}+0\vect{v}$인데 좌변은 $0\vect{v}$이므로
$0\vect{v}=0\vect{v}+0\vect{v}$. 양변에 $-(0\vect{v})$를 더하면 $0\vect{v}=\vect{0}$.

(2) $a(\vect{0}+\vect{0})=a\vect{0}+a\vect{0}$에서 좌변이 $a\vect{0}$이므로
같은 방식으로 $a\vect{0}=\vect{0}$.

(3) $\vect{v}+(-1)\vect{v}=(1+(-1))\vect{v}=0\vect{v}=\vect{0}$.
즉 $(-1)\vect{v}$는 $\vect{v}$의 역벡터이므로 $(-1)\vect{v}=-\vect{v}$.

(4) $(-a)\vect{v}+a\vect{v}=((-a)+a)\vect{v}=0\vect{v}=\vect{0}$.
따라서 $(-a)\vect{v}$는 $a\vect{v}$의 역벡터이므로 $(-a)\vect{v}=-(a\vect{v})$.
\end{proof}

\begin{example}[$\R^n$의 표준 벡터공간]
벡터를
\[
\vect{x}=(x_1,\dots,x_n),\quad \vect{y}=(y_1,\dots,y_n)
\]
로 두고
\[
\vect{x}+\vect{y}=(x_1+y_1,\dots,x_n+y_n),\quad
a\vect{x}=(ax_1,\dots,ax_n)
\]
로 정의하면 벡터공간 공리 8개가 모두 성립한다. 따라서 $\R^n$은 $\R$ 위의 벡터공간이다.
\end{example}

\begin{example}[다항식 공간 $P_n(\R)$]
\[
P_n(\R)=\{a_0+a_1x+\cdots+a_nx^n\mid a_i\in\R\}
\]
는 차수가 $n$ 이하인 실계수 다항식 전체의 집합이다. 다항식의 덧셈과 스칼라곱에 대해 닫혀 있으므로
$P_n(\R)$도 $\R$ 위의 벡터공간이다.
\end{example}

\begin{remark}
$\R^2$는 표준 연산으로는 $\R$ 위의 벡터공간이다. 이때 스칼라곱이 실수배로 정의되어 있으므로,
같은 연산을 그대로 둔 채 스칼라 체만 $\C$로 바꾸는 것은 불가능하다.
\end{remark}

\paragraph{Level 1. 기초 확인}
\begin{exercise}
$\Q,\Z,\R,\C$ 중 체인 것을 모두 고르시오.
\end{exercise}
\textbf{힌트.} 0이 아닌 원소의 곱셈 역원이 항상 존재하는지 확인하라.

\begin{exercise}
벡터공간 공리 중 분배법칙에 해당하는 두 식을 정확히 쓰시오.
\end{exercise}
\textbf{힌트.} 스칼라가 벡터합에 분배되는 식과, 스칼라합이 벡터에 분배되는 식을 구분하라.

\paragraph{Level 2. 표준 응용}
\begin{exercise}
$\W=\{(x,y,z)\in\R^3\mid x+y+z=0\}$가 $\R^3$의 부분공간인지 판정하시오.
\end{exercise}
\textbf{힌트.} 영벡터 포함, 덧셈 닫힘, 스칼라곱 닫힘의 세 조건을 점검하라.

\begin{exercise}
정리의 항목 (2)인 $a\vect{0}=\vect{0}$을 공리만으로 다시 증명하시오.
\end{exercise}
\textbf{힌트.} $\vect{0}+\vect{0}=\vect{0}$에서 시작하라.

\paragraph{Level 3. 연결 추론}
\begin{exercise}
벡터공간 $V$에서
\[
a(\vect{u}-\vect{v})=a\vect{u}-a\vect{v}
\]
를 증명하시오.
\end{exercise}
\textbf{힌트.} $\vect{u}-\vect{v}=\vect{u}+(-\vect{v})$로 바꾸고 $(-a)\vect{v}=-(a\vect{v})$를 사용하라.

\begin{exercise}
$\vect{u}+\vect{w}_1=\vect{v}+\vect{w}_2$이면 $\vect{u}-\vect{v}=\vect{w}_2-\vect{w}_1$임을 보여라.
\end{exercise}
\textbf{힌트.} 양변에 $-\vect{v}$와 $-\vect{w}_1$을 순서대로 더하라.

\paragraph{Level 4. 도전 확장}
\begin{exercise}
$\R^2$에서 덧셈은 표준 덧셈으로 두고 스칼라곱을
\[
a\odot(x,y)=(ax,y)
\]
로 정의하자. 이 구조가 $\R$ 위의 벡터공간인지 판정하시오.
\end{exercise}
\textbf{힌트.} $(a+b)\vect{v}=a\vect{v}+b\vect{v}$ 또는 $1\vect{v}=\vect{v}$를 대입해 보라.

\begin{exercise}
체 공리에서 곱셈 교환법칙을 제거하면 어떤 대수 구조가 되는지 조사하고,
선형대수에서 어떤 주제로 확장되는지 간단히 정리하시오.
\end{exercise}
\textbf{힌트.} division ring(또는 skew field) 키워드를 참고하라.

\begin{itemize}
  \item 벡터공간의 스칼라는 체 $K$에서 온다.
  \item 벡터공간 공리는 덧셈 구조와 스칼라곱 호환성을 동시에 요구한다.
  \item 공리만으로도 $0\vect{v}=\vect{0}$, $(-1)\vect{v}=-\vect{v}$ 같은 기본 법칙을 증명할 수 있다.
  \item $\R^n$, $P_n(\R)$는 대표적인 벡터공간이다.
  \item 이후의 기저, 차원, 선형사상 이론은 이 공리 체계 위에서 전개된다.
\end{itemize}

\sectionoutline{부분공간과 생성부분공간}
\sectionoutline{선형결합, 선형독립, 생성집합}
\sectionoutline{기저(Basis)와 좌표표현}
\sectionoutline{차원(Dimension)과 기저교환법}
\sectionoutline{부분공간의 합, 교집합, 직합}
\chapterstandaloneend
