\chapter{Jordan Canonical Form}

\chapterintro{이 장에서는 선형연산자의 구조를 Jordan 표준형으로 정리합니다. 존재성과 유일성을 엄밀히 확인한 뒤, 실제 계산 절차까지 연결하겠습니다.}
\chaptergoals{Jordan 블록과 사슬을 정의하고 계산할 수 있다.}{Jordan 형식 존재/유일성 정리를 설명할 수 있다.}{Jordan 형식을 반복작용 계산에 적용할 수 있다.}

\sectiontemplate{Jordan 블록, 존재성과 유일성}{Jordan 형식은 대각화가 실패한 경우를 가장 표준적으로 기술하는 틀입니다.}{\item Jordan 블록을 정의할 수 있습니다.\item 존재/유일성의 의미를 설명할 수 있습니다.\item 블록 크기와 불변량의 연결을 말할 수 있습니다.}{\item 일반화 고유벡터\item primary/cyclic decomposition}

\begin{definition}
$\lambda\in K$와 $k\in\mathbb{N}$에 대해
$$J_k(\lambda)=
\begin{bmatrix}
\lambda&1&0&\cdots&0\\
0&\lambda&1&\ddots&\vdots\\
\vdots&\ddots&\ddots&\ddots&0\\
\vdots&&\ddots&\lambda&1\\
0&\cdots&\cdots&0&\lambda
\end{bmatrix}
$$
를 Jordan 블록이라 한다.
\end{definition}

\begin{theorem}[Jordan 존재정리]
대수적으로 닫힌 체 위 유한차원 공간의 모든 선형연산자는 적절한 기저에서 Jordan 표준형으로 표현된다.
\end{theorem}

\begin{proof}
primary decomposition으로 고유값별 성분으로 나눈 뒤, 각 성분에서 nilpotent 부분에 대해 cyclic decomposition을 적용하면 Jordan 사슬 기저를 얻는다. 이들을 합치면 Jordan 형식이 된다.
\end{proof}

\begin{theorem}[유일성(블록 순서 제외)]
같은 연산자의 Jordan 형식은 블록 배열 순서를 제외하면 유일하다.
\end{theorem}

\begin{proof}
$\dim\ker (T-\lambda I)^j$의 증가량이 블록 크기 분포를 결정한다. 이 값들은 similarity 불변량이므로 블록 분포가 유일하게 정해진다.
\end{proof}

\subsection*{주의}
\begin{warning}
Jordan 형식의 유일성은 "행렬 그대로"가 아니라 "블록 다중집합"의 유일성입니다.
\end{warning}
\begin{warning}
체가 대수적으로 닫히지 않으면 Jordan 형식 대신 rational canonical form이 필요할 수 있습니다.
\end{warning}

\subsection*{자가진단퀴즈}
\begin{enumerate}[label=\arabic*.]
\item $J_3(2)$의 최소다항식과 특성다항식을 구하십시오.
\item 블록 크기와 $\dim\ker(T-\lambda I)^j$의 관계를 설명하십시오.
\item 대각화 가능성과 Jordan 블록 크기의 관계를 서술하십시오.
\end{enumerate}

\sectiontemplate{Jordan 계산과 응용}{실제 계산에서는 사슬을 어떻게 찾는지가 핵심입니다.}{\item Jordan 형식 계산 절차를 재현할 수 있습니다.\item Jordan 기저를 구성할 수 있습니다.\item $A^n$, $e^{tA}$ 계산에 적용할 수 있습니다.}{\item kernel 사슬\item 행렬함수}

\begin{example}
$$A=\begin{bmatrix}2&1\\0&2\end{bmatrix}=J_2(2)$$
이므로
$$A^n=\begin{bmatrix}2^n&n2^{n-1}\\0&2^n\end{bmatrix}$$
입니다.
\end{example}

\subsection*{주의}
\begin{warning}
Jordan 사슬 계산에서 벡터 선택을 임의로 바꾸면 이전 단계 조건이 깨질 수 있습니다. 단계별 조건을 기록하십시오.
\end{warning}
\begin{warning}
계산 중에 고유값별 블록을 섞으면 검산이 어려워집니다. 고유값별로 분리해 작업하십시오.
\end{warning}

\subsection*{자가진단퀴즈}
\begin{enumerate}[label=\arabic*.]
\item $3\times3$ 비대각화 행렬 하나를 택해 Jordan 형식을 구하십시오.
\item Jordan 형식으로 $A^n$을 계산하는 절차를 써 보십시오.
\item $e^{tJ_k(\lambda)}$의 형태를 설명하십시오.
\end{enumerate}

\chapterapplicationtemplate{Jordan 형식을 이용해 상수계수 선형미분방정식 $x'(t)=Ax(t)$의 해 형태를 분류해 보십시오.}

\chaptersummarytemplate
\chapterexercisestemplate
