\chapter{행렬표현, 기저변환, Similarity}

\chapterintro{같은 선형사상이라도 기저를 바꾸면 행렬표현이 달라집니다. 이 장에서는 "본질은 같고 표현만 다르다"는 관점을 수식으로 고정하겠습니다.}
\chaptergoals{선형사상의 행렬표현을 구성할 수 있다.}{기저변환 공식을 유도하고 적용할 수 있다.}{similarity와 불변량의 의미를 설명할 수 있다.}

\sectiontemplate{선형사상의 행렬표현}{기저가 정해지면 선형사상은 행렬로 완전히 기술됩니다.}{\item 표현행렬을 구성할 수 있습니다.\item 합성과 행렬곱의 대응을 설명할 수 있습니다.\item 좌표벡터 관점을 사용할 수 있습니다.}{\item 기저\item 선형사상}

\begin{definition}
기저 $\mathcal{B}=(v_1,\dots,v_n)$, $\mathcal{C}=(w_1,\dots,w_m)$가 주어졌을 때 $T:V\to W$의 \emph{표현행렬} $[T]_{\mathcal{C}\leftarrow\mathcal{B}}$는
$$[T(v_j)]_{\mathcal{C}}$$
를 $j$번째 열로 갖는 행렬이다.
\end{definition}

\paragraph{증명 전략.}
벡터 $x$를 기저좌표로 쓰고 $T(x)$의 좌표를 비교하면 표현행렬이 선형사상 계산을 대체함을 보일 수 있습니다.

\begin{theorem}
모든 $x\in V$에 대해
$$[T(x)]_{\mathcal{C}}=[T]_{\mathcal{C}\leftarrow\mathcal{B}}[x]_{\mathcal{B}}$$
가 성립한다.
\end{theorem}

\begin{proof}
$x=\sum_j \alpha_j v_j$라 두면 선형성으로
$$T(x)=\sum_j \alpha_jT(v_j).$$
양변의 $\mathcal{C}$-좌표를 취하면 우변은 표현행렬의 열결합이므로 식이 성립한다.
\end{proof}

\paragraph{의미 해석.}
선형사상 계산은 기저를 고정하면 행렬곱 계산으로 환원됩니다.

\subsection*{주의}
\begin{warning}
$[T]_{\mathcal{C}\leftarrow\mathcal{B}}$에서 출발 기저와 도착 기저의 순서를 바꾸면 전혀 다른 행렬이 됩니다.
\end{warning}
\begin{warning}
기저를 바꾸면 벡터 좌표와 사상 행렬이 동시에 변합니다.
\end{warning}

\subsection*{자가진단퀴즈}
\begin{enumerate}[label=\arabic*.]
\item 주어진 기저에서 선형사상의 행렬표현을 계산해 보십시오.
\item 왜 열벡터가 $T(v_j)$의 좌표가 되는지 설명하십시오.
\item 합성사상 행렬공식을 증명하십시오.
\end{enumerate}

\sectiontemplate{기저변환과 Similarity}{기저변환은 같은 사상의 다른 표현입니다. similarity는 이 관계를 정리한 등가관계입니다.}{\item 기저변환행렬을 구성할 수 있습니다.\item similarity 공식을 유도할 수 있습니다.\item similarity 불변량을 설명할 수 있습니다.}{\item 역행렬\item 표현행렬}

\paragraph{증명 전략.}
한 기저에서 다른 기저로 좌표를 옮기는 행렬 $P$를 도입하고, 좌표식 두 개를 합성해 $A'=P^{-1}AP$를 얻습니다.

\begin{theorem}
같은 선형사상 $T:V\to V$의 두 표현행렬 $A,A'$에 대해 어떤 가역행렬 $P$가 존재하여
$$A'=P^{-1}AP$$
를 만족한다.
\end{theorem}

\begin{proof}
기저 $\mathcal{B},\mathcal{B}'$에 대해 좌표변환이 $[x]_{\mathcal{B}}=P[x]_{\mathcal{B}'}$라 하자. 그러면
$$[T(x)]_{\mathcal{B}'}=P^{-1}[T(x)]_{\mathcal{B}}=P^{-1}A[x]_{\mathcal{B}}=P^{-1}AP[x]_{\mathcal{B}'}.$$
또한 좌변은 $A'[x]_{\mathcal{B}'}$이므로 $A'=P^{-1}AP$.
\end{proof}

\paragraph{의미 해석.}
행렬은 바뀌어도 선형사상의 본질은 바뀌지 않습니다. 고유값, 행렬식, 트레이스 같은 양이 불변량으로 남습니다.

\subsection*{주의}
\begin{warning}
닮음(similarity)과 합동(congruence)은 다른 관계입니다. 문맥을 섞지 마십시오.
\end{warning}
\begin{warning}
$P^{-1}AP$ 순서를 $PAP^{-1}$로 바꾸면 일반적으로 틀립니다.
\end{warning}

\subsection*{자가진단퀴즈}
\begin{enumerate}[label=\arabic*.]
\item 유사행렬이 같은 특성다항식을 가짐을 보이십시오.
\item $A'=P^{-1}AP$에서 $\det A'=\det A$를 확인하십시오.
\item 같은 사상인데 행렬이 다른 간단한 예를 하나 제시하십시오.
\end{enumerate}

\chapterapplicationtemplate{좌표계가 다른 두 관측 프레임에서 같은 선형변환이 어떻게 다른 행렬로 기록되는지 계산해 보십시오.}

\chaptersummarytemplate
\chapterexercisestemplate
