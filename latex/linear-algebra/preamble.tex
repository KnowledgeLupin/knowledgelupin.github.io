\usepackage{kotex}
\usepackage{amsmath,amssymb,amsthm,mathtools}
\usepackage[margin=1in]{geometry}
\usepackage{hyperref}
\usepackage{bookmark}
\usepackage{enumitem}
\usepackage{xcolor}
\usepackage{titlesec}

\setlist[itemize]{leftmargin=2.2em}
\setlist[enumerate]{leftmargin=2.5em}

\hypersetup{
  colorlinks=true,
  linkcolor=blue!50!black,
  urlcolor=blue!50!black,
  citecolor=blue!50!black
}

\newtheorem{definition}{정의}[section]
\newtheorem{theorem}[definition]{정리}
\newtheorem{proposition}[definition]{명제}
\newtheorem{example}[definition]{예제}
\newtheorem{exercise}[definition]{연습문제}

\theoremstyle{remark}
\newtheorem*{remark}{Remark}

\newcommand{\R}{\mathbb{R}}
\newcommand{\C}{\mathbb{C}}
\newcommand{\Q}{\mathbb{Q}}
\newcommand{\Z}{\mathbb{Z}}
\newcommand{\V}{\mathcal{V}}
\newcommand{\W}{\mathcal{W}}
\newcommand{\vect}[1]{\mathbf{#1}}

\newcommand{\sectionoutline}[1]{%
\section{#1}
\subsection*{동기}
이 절은 작성 예정입니다.

\subsection*{정의}
핵심 정의를 작성 예정입니다.

\subsection*{정리와 증명}
핵심 정리와 증명을 작성 예정입니다.

\subsection*{예제}
기초 예제와 연결 예제를 작성 예정입니다.

\subsection*{연습문제 (4단계 + 힌트)}
Level 1--4 문제와 힌트를 작성 예정입니다.

\subsection*{요약}
핵심 요약 5줄을 작성 예정입니다.
}
