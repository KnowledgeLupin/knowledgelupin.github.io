\usepackage{kotex}
\usepackage{fontspec}
\usepackage{amsmath,amsthm,mathtools}
\usepackage{unicode-math}
\usepackage[margin=1in]{geometry}
\usepackage{hyperref}
\usepackage{bookmark}
\usepackage{enumitem}
\usepackage{xcolor}
\usepackage{titlesec}

% Text fonts: preset B (modern sans)
\setmainfont{Helvetica Neue}
\setsansfont{Helvetica Neue}
\setmonofont{Menlo}
\setmainhangulfont[
  AutoFakeSlant=0.18
]{Apple SD Gothic Neo}
\setsanshangulfont[
  AutoFakeSlant=0.18
]{Apple SD Gothic Neo}

% Math font: preset A
\setmathfont{STIX Two Math}

\setlist[itemize]{leftmargin=2.2em}
\setlist[enumerate]{leftmargin=2.5em}

\hypersetup{
  colorlinks=true,
  linkcolor=blue!50!black,
  urlcolor=blue!50!black,
  citecolor=blue!50!black
}

\newtheorem{definition}{정의}[section]
\newtheorem{theorem}[definition]{정리}
\newtheorem{proposition}[definition]{명제}
\newtheorem{example}[definition]{예제}
\newtheorem{exercise}{연습문제}[section]
\newtheorem{warning}[definition]{주의}
\newtheorem{application}[definition]{응용}

\theoremstyle{remark}
\newtheorem*{remark}{비고}

\newcommand{\R}{\mathbb{R}}
\newcommand{\C}{\mathbb{C}}
\newcommand{\Q}{\mathbb{Q}}
\newcommand{\Z}{\mathbb{Z}}
\newcommand{\W}{\mathcal{W}}
\newcommand{\vect}[1]{\mathbf{#1}}

\newcommand{\sectionoutline}[1]{%
\section{#1}
이 절은 원고 작성 예정입니다.
}

\newcommand{\chapterintro}[1]{%
\section*{장 도입}
#1
}

\newcommand{\chaptergoals}[3]{%
\section*{장 학습목표}
\begin{enumerate}[label=\arabic*.]
\item #1
\item #2
\item #3
\end{enumerate}
}

\newcommand{\sectiontemplate}[4]{%
\section{#1}
\subsection*{도입 설명}
#2
\subsection*{학습목표}
\begin{enumerate}[label=\arabic*.]
#3
\end{enumerate}
\subsection*{선수개념 체크}
\begin{itemize}
#4
\end{itemize}
\subsection*{핵심 내용}
}

\newcommand{\chapterapplicationtemplate}[1]{%
\section{응용}
\begin{application}
#1
\end{application}
}

\newcommand{\chaptersummarytemplate}{%
\section*{장 요약}
\begin{itemize}
\item 핵심 정의 5개, 핵심 정리 5개, 연결 문장 5개를 작성합니다.
\item 이 장에서 다음 장으로 넘어가기 위한 의존 개념을 명시합니다.
\end{itemize}
}

\newcommand{\chapterexercisestemplate}{%
\section*{종합 연습문제}
\begin{itemize}
\item 연습문제 A: 기초 확인 6문항 이상
\item 연습문제 B: 개념 연결 4문항 이상
\item 연습문제 C: 증명/종합 2문항 이상
\end{itemize}
}
