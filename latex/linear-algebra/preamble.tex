\usepackage{kotex}
\usepackage{fontspec}
\usepackage{amsmath,amsthm,mathtools}
\usepackage{unicode-math}
\usepackage[margin=1in]{geometry}
\usepackage{hyperref}
\usepackage{bookmark}
\usepackage{enumitem}
\usepackage{xcolor}
\usepackage{titlesec}

% Text fonts: preset B (modern sans)
\setmainfont{Helvetica Neue}
\setsansfont{Helvetica Neue}
\setmonofont{Menlo}
\setmainhangulfont[
  AutoFakeSlant=0.18
]{Apple SD Gothic Neo}
\setsanshangulfont[
  AutoFakeSlant=0.18
]{Apple SD Gothic Neo}

% Math font: preset A
\setmathfont{STIX Two Math}

\setlist[itemize]{leftmargin=2.2em}
\setlist[enumerate]{leftmargin=2.5em}

\hypersetup{
  colorlinks=true,
  linkcolor=blue!50!black,
  urlcolor=blue!50!black,
  citecolor=blue!50!black
}

\newtheorem{definition}{정의}[section]
\newtheorem{theorem}[definition]{정리}
\newtheorem{proposition}[definition]{명제}
\newtheorem{example}[definition]{예제}
\newtheorem{exercise}{연습문제}[section]

\theoremstyle{remark}
\newtheorem*{remark}{Remark}

\newcommand{\R}{\mathbb{R}}
\newcommand{\C}{\mathbb{C}}
\newcommand{\Q}{\mathbb{Q}}
\newcommand{\Z}{\mathbb{Z}}
\newcommand{\W}{\mathcal{W}}
\newcommand{\vect}[1]{\mathbf{#1}}

\newcommand{\sectionoutline}[1]{%
\section{#1}
이 절은 원고 작성 예정입니다.
}
